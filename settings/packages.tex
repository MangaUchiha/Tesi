% Add your own packages here, 
% these existing packages can be removed if necessary
% note that more packages are imported in Thesis.cls, '
% but those should not be changed if you don't know what you are doing...

\usepackage[utf8]{inputenc} % for writing other that basic characters
\usepackage{lmodern} % to remove restriction on font size
\usepackage{textcomp} % to insert euro sign faking th font
\RequirePackage{fix-cm} % to remove restriction on font size
\usepackage{graphicx}
\graphicspath{ {images/} }
\usepackage{caption}
\usepackage{subcaption}

\usepackage{float} %These 2 enable "H" and can avoid tables from floating 
%\restylefloat{table} %Commented because captions disappear
\usepackage{longtable} % Use multipage tables
\usepackage{array} %Include stuff
\usepackage{tabularx, ragged2e} % For defining table width
\newcolumntype{C}[1]{>{\centering\arraybackslash}p{#1}} % New column type for centered and fill column

\usepackage{enumitem}

\usepackage[usenames, dvipsnames]{xcolor} %Including more colors

\usepackage[super]{nth} %Can use th on top

\usepackage{pgfplots}  % Include plots and graphs
\pgfplotsset{compat=1.14} %Set pgfplot versions
\usetikzlibrary{pgfplots.statistics} %Include boxplots and histograms

\usepackage{siunitx} % Used to write the mu letter

\usepackage{multirow} %Enables multiwors

\usepackage[bottom]{footmisc} % Sticks footnotes to the bottom of the page

\hyphenation{im-pie-ga-to} %Specific hyphenation

% Include any extra LaTeX packages required
\usepackage[square, numbers, comma, sort&compress]{natbib}  % Use the "Natbib" style for the references in the Bibliography
\usepackage{verbatim}  		% Needed for the "comment" environment to make LaTeX comments
\usepackage{hyperref}		% Clickable links
\usepackage{csquotes}		% Multiline quotes
\usepackage{pgfgantt}		% Gantt charts
\usepackage{listings}		% For nice code
\usepackage{color} 			% For defining custom colors
\usepackage{float} 			% For fixed-place figures
\usepackage{easytable}		% For more flexible tables
\usepackage{multirow}		% For multirow options in tables
\usepackage{colortbl} 		% For coloring tables
\usepackage[htt]{hyphenat}	% Enable wordbreak on tt text
\usepackage{xcolor}			% For coloring enumerate number
\usepackage{geometry,calc}	% To expand margins

% Configurations
\setlength{\paperheight}{297mm}
\setlength{\paperwidth}{210mm}
\setlength{\textwidth}{\paperwidth-60mm}
\setlength{\textheight}{\paperheight-60mm-\headheight-\headsep-\footskip}
\setlength{\topmargin}{3cm-1in}
\setlength{\oddsidemargin}{3cm-1in}
\setlength{\evensidemargin}{0in}
\setlength{\hoffset}{0in}
\setlength{\voffset}{0in}

\hyphenation{Para-View}	% Defines ParaView's preferred hyphenation
\hyphenation{Para-Unity}	% Defines ParaView's preferred hyphenation


\definecolor{codebackground}{rgb}{.95,.95,.95}
\definecolor{dkgreen}{rgb}{0,0.6,0}
\definecolor{gray}{rgb}{0.5,0.5,0.5}
\definecolor{mauve}{rgb}{0.58,0,0.82}
\definecolor{tblgreen}{rgb}{0.45,1,0.84}
\definecolor{tblorange}{rgb}{1,0.83,0.47}
\definecolor{tblpink}{rgb}{1,0.54,0.85}

\hypersetup{
	colorlinks,
	citecolor=black,
	filecolor=black,
	linkcolor=black,
	urlcolor=black
}

\newcommand{\source}[1]{\vspace{-3pt} \caption*{\textit{Source: {#1}}} }	% For citing sources in captions

\newcommand*{\doi}[1]{\texttt{doi: \href{http://dx.doi.org/#1}{#1}}} % For DOIs in bibliography

% Enumerate with custom colors
\let\svitem\item
\newenvironment{cenumerate}[1][\relax]{\renewcommand\item[1][black]{\color{##1}\svitem}
	\ifx\relax#1\enumerate\else\enumerate[#1]\fi}{\endenumerate}


% Listings ---------------------
% Old exmaple
%\lstset{language=[Sharp]C,
%	basicstyle=\ttfamily\scriptsize,
%	backgroundcolor=\color{codebackground},
%	keywordstyle=\color{blue}\ttfamily,
%	stringstyle=\color{red}\ttfamily,
%	commentstyle=\color{dkgreen}\ttfamily,
%	stringstyle=\color{mauve}\ttfamily,
%	morecomment=[l][\color{dkgreen}]{\#},
% breaklines=true,
%	tabsize=2,
%	showspaces=false,
%	showstringspaces=false,
%	aboveskip=2em,
%	belowskip=2em,
	% frame=none,
%	postbreak=\raisebox{0ex}[0ex][0ex]{\ensuremath{\color{red}\hookrightarrow\space}}
%}

\lstset{language=[Visual]Basic,
	basicstyle=\ttfamily\scriptsize,
	backgroundcolor=\color{codebackground},
	keywordstyle=\color{blue}\ttfamily,
	stringstyle=\color{red}\ttfamily,
	commentstyle=\color{dkgreen}\ttfamily,
	stringstyle=\color{mauve}\ttfamily,
	morecomment=[l][\color{dkgreen}]{\#},
	breaklines=true,
	tabsize=2,
	showspaces=false,
	showstringspaces=false,
	aboveskip=2em,
	belowskip=2em,
	% frame=none,
	postbreak=\raisebox{0ex}[0ex][0ex]{\ensuremath{\color{red}\hookrightarrow\space}}
}