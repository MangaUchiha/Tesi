%*******10********20********30********40********50********60********70********80

% For all chapters, use the newdefined chap{} instead of chapter{}
% This will make the text at the top-left of the page be the same as the chapter

\chap{Discussion}\label{chap:discussion}
In the following and last chapter of this thesis, the results presented in Chapter \ref{chap:results} will be discussed. We'll focus on describing the most common diseases among internationally adopted children (in Section \ref{sec:healthstatusofIAC}), analyzing which tests are in our opinion worthwhile amongst the ones listed in the GNLBI protocol (in Section \ref{sec:worthwhile?}), and lastly present our conclusions on the most relevant statistical correlations we found (in Section \ref{sec:seekingforsupport}). 

\section{Health status of internationally adopted children}\label{sec:healthstatusofIAC}
We found that statural and ponderal delayed growth are fairly common among international adoptees. These alterations are probably complex and multifactorial. We did not find statistical evidence for any possible cause, but we believe that FASD (that wasn't considered in this study), malnutrition, anemia, tuberculosis and gastrointestinal parasitic infections are the most probable major factors in determining this growth delay. Literature shows that these children are \textit{late bloomers} and they will grow up to their range height and weight during adolescence.

Then, anemia was found to be another frequent condition. It resulted linked to iron-deficiency for the most part, enforcing the need for iron supplements. The other leading cause is $\beta$-thalassemia minor, which strongly requires serum protein electrophoresis, alongside a complete blood count. On the other hand, our study evidenced that gastrointestinal parasitic infections, despite being very common, are not a common cause on anemia in internationally adopted children, as instead modern literature and common knowledge would suggest.

As for the immunization status, this often offers an incomplete protection from vaccine-preventable diseases (65\% of all children) and, therefore, it should be strongly advised to screen all internationally adopted children for their immunization status at their arrival in each country of destination,. We concur with international guidelines that pre-adoptive immunization records may not be assumed as truthful or correct.

Moreover, vitamin D serum levels are commonly deficient or insufficient in international adoptees (40\%). This consideration strongly reinforces the need for vitamin D prophylaxis  in this special population, perhaps even without testing for vitamin D status, if the clinical evaluation that suggest rickets.

Lastly, infectious diseases are common among international adoptees, but these are rarely severe: HIV, HBV, HCV and syphilis are almost never present: 1, 4, 0 and 1 cases respectively in our 17-year-spanning study. On the other hand latent tuberculosis (16.84\%) and gastrointestinal parasitic infection (41.22\%) are fairly common.

In the following section \ref{sec:worthwhile?}, we'll present our recommendations in light of these new findings.

\section{Is everything we do worthwhile?}\label{sec:worthwhile?}
Lorem ipsum dolor sit amet, consectetur adipiscing elit. Pellentesque nibh metus, suscipit a scelerisque sit amet, rhoncus et lectus. Mauris eget erat rutrum, euismod massa id, maximus mauris. Nulla maximus, ex sit amet lacinia consequat, enim ante mollis dui, sit amet tincidunt massa felis id magna. Aenean gravida ante nec volutpat rutrum. Cras eget ullamcorper leo. Curabitur eu volutpat tellus. Integer nec ornare sapien. Fusce ipsum justo, interdum quis libero a, mattis tristique velit. Phasellus rhoncus lorem non ultrices luctus.

\subsection{What we believe to be worthwhile}\label{sub:worthwhile}
Lorem ipsum dolor sit amet, consectetur adipiscing elit. Pellentesque nibh metus, suscipit a scelerisque sit amet, rhoncus et lectus. Mauris eget erat rutrum, euismod massa id, maximus mauris. Nulla maximus, ex sit amet lacinia consequat, enim ante mollis dui, sit amet tincidunt massa felis id magna. Aenean gravida ante nec volutpat rutrum. Cras eget ullamcorper leo. Curabitur eu volutpat tellus. Integer nec ornare sapien. Fusce ipsum justo, interdum quis libero a, mattis tristique velit. Phasellus rhoncus lorem non ultrices luctus.
% Mantoux
% Emocromo con MCV, ferritina e protidogramma
% Immunization status
%E Questo è in linea con la letteratura?

\subsection{And what may not be}\label{sub:notworthwhile}
Lorem ipsum dolor sit amet, consectetur adipiscing elit. Pellentesque nibh metus, suscipit a scelerisque sit amet, rhoncus et lectus. Mauris eget erat rutrum, euismod massa id, maximus mauris. Nulla maximus, ex sit amet lacinia consequat, enim ante mollis dui, sit amet tincidunt massa felis id magna. Aenean gravida ante nec volutpat rutrum. Cras eget ullamcorper leo. Curabitur eu volutpat tellus. Integer nec ornare sapien. Fusce ipsum justo, interdum quis libero a, mattis tristique velit. Phasellus rhoncus lorem non ultrices luctus.
% Glicemia, Creatinina,VES e Urine perché?
% No Vitamin D
% No Parassiti
% No HCV, HBV, Lue (TPHA), HIV 1 e 2
%E Questo è in linea con la letteratura?

\section{Seeking for support: building correlations}\label{sec:seekingforsupport}
Lorem ipsum dolor sit amet, consectetur adipiscing elit. Pellentesque nibh metus, suscipit a scelerisque sit amet, rhoncus et lectus. Mauris eget erat rutrum, euismod massa id, maximus mauris. Nulla maximus, ex sit amet lacinia consequat, enim ante mollis dui, sit amet tincidunt massa felis id magna. Aenean gravida ante nec volutpat rutrum. Cras eget ullamcorper leo. Curabitur eu volutpat tellus. Integer nec ornare sapien. Fusce ipsum justo, interdum quis libero a, mattis tristique velit. Phasellus rhoncus lorem non ultrices luctus.

\section{Objectives achieved}\label{secd:objectivesachieved}
Lorem ipsum dolor sit amet, consectetur adipiscing elit. Pellentesque nibh metus, suscipit a scelerisque sit amet, rhoncus et lectus. Mauris eget erat rutrum, euismod massa id, maximus mauris. Nulla maximus, ex sit amet lacinia consequat, enim ante mollis dui, sit amet tincidunt massa felis id magna. Aenean gravida ante nec volutpat rutrum. Cras eget ullamcorper leo. Curabitur eu volutpat tellus. Integer nec ornare sapien. Fusce ipsum justo, interdum quis libero a, mattis tristique velit. Phasellus rhoncus lorem non ultrices luctus.

% Da tenere se ho tempo (hehe) e se le conclusioni non sono palesi nei paragrafi sopra e quindi qui non sarebbero ridondanti.
%COSA SIAMO RIUSCITI A CAPIRE
%COSA SI DOVREBBE FARE QUINDI

\section{Future work}\label{sec:futurework}
Lorem ipsum dolor sit amet, consectetur adipiscing elit. Pellentesque nibh metus, suscipit a scelerisque sit amet, rhoncus et lectus. Mauris eget erat rutrum, euismod massa id, maximus mauris. Nulla maximus, ex sit amet lacinia consequat, enim ante mollis dui, sit amet tincidunt massa felis id magna. Aenean gravida ante nec volutpat rutrum. Cras eget ullamcorper leo. Curabitur eu volutpat tellus. Integer nec ornare sapien. Fusce ipsum justo, interdum quis libero a, mattis tristique velit. Phasellus rhoncus lorem non ultrices luctus.

%QUI INSERISCI COSA SI POTREBBE ANCORA FARE
%O MAGARI UN DATABASE NAZIONALE CON I RISULTATI DELLE ANALISI per studi epidemiologici approfonditi o ingradire la
%popolazione per più correlazioni che a noi sono sfuggite a causa dei pochi dati.