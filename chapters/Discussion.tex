%*******10********20********30********40********50********60********70********80

% For all chapters, use the newdefined chap{} instead of chapter{}
% This will make the text at the top-left of the page be the same as the chapter

\chap{Discussion}\label{chap:discussion}
In the following and last chapter of this thesis, the results presented in Chapter \ref{chap:results} will be discussed. We'll focus on describing the most common diseases among internationally adopted children (in Section \ref{sec:healthstatusofIAC}), analyzing which tests are in our opinion worthwhile amongst the ones listed in the GNLBI protocol (in Section \ref{sec:worthwhile?}), and lastly present our conclusions on the most relevant statistical correlations we found (in Section \ref{sec:seekingforsupport}). 

\section{Health status of internationally adopted children}\label{sec:healthstatusofIAC}
We found that statural and ponderal delayed growth are fairly common among international adoptees. These alterations are probably complex and multifactorial. We did not find statistical evidence for any possible cause, but we believe that FASD (that wasn't considered in this study), malnutrition, anemia, tuberculosis and gastrointestinal parasitic infections are the most probable major factors in determining this growth delay. Literature shows that these children are \textit{late bloomers} and they will grow up to their range height and weight during adolescence.

Then, anemia was found to be another frequent condition. It resulted linked to iron-deficiency for the most part, enforcing the need for iron supplements. The other leading cause is $\beta$-thalassemia minor, which strongly requires serum protein electrophoresis, alongside a complete blood count. On the other hand, our study evidenced that gastrointestinal parasitic infections, despite being very common, are not a common cause on anemia in internationally adopted children, as instead modern literature and common knowledge would suggest.

As for the immunization status, this often offers an incomplete protection from vaccine-preventable diseases (65\% of all children) and, therefore, it should be strongly advised to screen all internationally adopted children for their immunization status at their arrival in each country of destination,. We concur with international guidelines that pre-adoptive immunization records may not be assumed as truthful or correct.

Moreover, vitamin D serum levels are commonly deficient or insufficient in international adoptees (40\%). This consideration strongly reinforces the need for vitamin D prophylaxis  in this special population, perhaps even without testing for vitamin D status, if the clinical evaluation that suggest rickets.

Lastly, infectious diseases are common among international adoptees, but these are rarely severe: HIV, HBV, HCV and syphilis are almost never present: 1, 4, 0 and 1 cases respectively in our 17-year-spanning study. On the other hand latent tuberculosis (16.84\%) and gastrointestinal parasitic infection (41.22\%) are fairly common.

In the following section \ref{sec:worthwhile?}, we'll present our recommendations in light of these new findings.

\section{Is everything we do worthwhile?}\label{sec:worthwhile?}
The \textit{GLNBI diagnostic-aiding protocol} (found in \cite{GNLBI2}) is a milestone for adoption-specialized pediatricians. Although it's thorough and complete, it may be too extensive for a universally applied screening protocol, and not all of the recommended tests and examinations are useful in a cost-benefit perspective. In particular, some sections do not directly influence the operative clinical path the child must follow and are therefore suppressible. Others, on the other hand, have a very strong impact on the child's health and strongly influence the further development of child's clinical plan.

In the following Sections \ref{sub:worthwhile} and \ref{sub:notworthwhile}, our considerations about the different segments of the protocol will be presented, alongside our recommendations.

\subsection{What we believe to be worthwhile}\label{sub:worthwhile}
First of all, performing a complete blood count is one of the most useful laboratory test, in a cost-benefit perspective. The measurement of MCV, ferritin serum levels, and of the distribution of serum protein via electrophoresis completes the evaluation of anemia and of its most frequent etiologies. These tests are necessary and pediatricians should be strongly advised towards performing them as soon as possible in internationally adopted children.\\
In our study, 280 blood counts and 243 serum protein electrophoresis were performed, 251 serum ferritin levels were tested. These costed respectively: 5.30 €, 7.00 €, and 13.90 €. Total costs amount to: 1484.00 €, 1701.00 € 3488.90 € respectively. The grand total amounted to 6,673.90 € for a complete evaluation of anemic states. We believe this to be a totally reasonable expense because of the high impact it has on the child's diagnostic and (especially) therapeutic operations.

As explained in \ref{sec:descriptiveanalyses} on page \pageref{sec:descriptiveanalyses}, the immunization status is long from acceptable in immigrant children: 65\% of our population resulted not entirely protected from vaccine-preventable diseases. Modern medical literature hasn't settled yet whether it's best to presume the child not be vaccinated at all and repeat all shots according to the national vaccination calendar, or to test for individual antibody serum levels and repeat vaccination shots only for the ones the child resulted exposed to. We prefer the second approach, because it spared the child of useless injections.\\
Each immunization status panel costs 70.70 € and we performed a total of 224 panels, for a total cost of 15.836,80 €. This set of tests  is clearly very expensive, but so are vaccinations. Thus, we assume this is a favorable expense, even if no calculations were performed to determine which approach would better employ resources.

Another worthwhile exam is tuberculosis testing via the Mantoux intradermal reaction test. This is performed by injecting PPD in the dermis (or corium) of the child and waiting for a local flogistical response in the next 24/48 hour period. This is the cheapest way of testing for tubercular exposition, since a dose of PPD only costs 7.00 €. We performed 273 Mantoux tests of the years for a total cost of 1,911.00 €: a low-priced evaluation immediately directing the patient to a chest X-ray or at home.

A complete overview of the tests' costs can be found in Table \ref{tab:costs}.

\subsection{And what may not be}\label{sub:notworthwhile}
Vitamin D deficiency and insufficiency are common among international adoptees. We tested 180 vitamin D serum levels (20.30 € each) and 72 turned out positive (40\%). Since vitamin D supplementation is harmless and that a three-month therapy period costs less than the single test, we advise towards eliminating the test from the screening protocol and administer cholecalciferol (vitamin D$_3$) to all internationally adopted children for 3 months in prophylaxis dosage.

Similarly to vitamin D, testing for parasitic infections could be labeled as redundant. All children are administered a dose of \textit{mebendazole} and one of \textit{tinidazole} just after the stool samples are collected. This is an approach we support, but it relegates parasite stool research to a test out of mere curiosity and should therefore be abandoned.\\
In our study, we performed 243 tests for parasites, of which 90 resulted positive, but none of these changed the children's therapeutical future, since they were administered doses of \textit{antihelmintics} both if the test ended up positive or not. Each test costs 13.50 € and its removal from the protocol would have saved a total of 3,240.00 €.

Since testing for HCV, HBV, syphilis via TPHA, and both genotypes of HIV resulted in little to no actual pathological evidence, we suggest that these exams shouldn't be included in early screening, due to the inconsistency of their benefits. Rather they should be performed only if clinical examination points towards such infections or anamnestic risk factors are identified.\\
We performed 215 tests for HBV, 282 fro HCV, 256 for syphiis, and 282 for HIV. 18,818.40 € would have been saved, if these analyses would have been performed only on positive children.

\section{Seeking for support: building correlations}\label{sec:seekingforsupport}
As explained in Section \ref{sec:statisticalinference} on page \pageref{sec:statisticalinference}, we strongly believed in the need for strong results from statistical inference, and, therefore, this was one of the main focus points of this study. As explained in \ref{sec:aim}, the research for correlations amongst the considered variables could have meant spare some hospitalization time and some invasive tests from the children brought to our attention, achieving benefits on an economical, health and comfort level.

Most of our findings in revolved around anemia.

\begin{enumerate}
	\item \textbf{Anemia and MCV}. This correlation enforces the knowledge of the high prevalence of iron-deficient anemia. The leading causes are: malnutrition and insufficient iron daily intake, and chronic gastrointestinal bleeding. The last option that must be considered in internationally adopted children is $\beta$-thalassemia minor.
	\item \textbf{Low-MCV anemia and ferritin}. This correlation strengthens even more the need for acknowledgment of the high prevalence of iron-deficient anemia. Moreover, it boldly calls for iron supplement-based therapy or prophylaxis in the internationally adopted child.
	\item \textbf{Low-MCV anemia and country of origin}. Regarding iron-deficient anemia, special attention should be payed to children coming from the Russian Federation and Bulgaria, in Eastern Europe, and India, in Asia.
	\item \textbf{Ferritin and group 1 gastrointestinal parasitic infection}. The finding justifies the need for parasite grouping and the little for need for them to be considered \textit{genus} by \textit{genus}.
\end{enumerate}

\section{Objectives achieved}\label{secd:objectivesachieved}
In Section \ref{sec:aim} on page \pageref{sec:aim}, the aim of the study was presented. These were achieved in period of seven months. We evaluated the most common diseases among internationally adopted children, we found which tests could be overabundant and found useful correlations that stress the strongest urges of internationally adopted children.

\section{Future work}\label{sec:futurework}
Despite us believing that many goals set for this study have been achieved, we also think that future work can be done in order to accomplish a better understanding of this particular pediatric population. Other studies with a larger population in exam could highlight correlations we weren't able to find. In particular, low height-for-age, \textit{genera}-specific parasitic infections, and tuberculosis seem have space for improvements in their diagnostic and therapeutic processes.

Moreover, the establishment of a national database could be the next step taken to allow all 20 Italian GLNBI Centers to contribute to further study which pathologies affect these children. National and international policies constantly alter infectious diseases' epidemiology, thus a dedicated and steady surveillance is mandatory, in order to allow at least some degree of plasticity in operative practices.  