%*******10********20********30********40********50********60********70********80

% For all chapters, use the newdefined chap{} instead of chapter{}
% This will make the text at the top-left of the page be the same as the chapter

\chap{Introduction}\label{chap:introduction}
Children's health has historically always been a sensitive and concerning matter for human kind. We can find reason for this in our universal instinctive draw towards the protection and care for our offspring, and in how children can be struck by some of the most devastating and life-wrecking diseases. Sometimes, these are the phenotypical expression of genetical marks, scarred onto and into these kids. Despite the origin, the color of the skin or the culture the child bears in his lineage, human beings feel the need to raise and safeguard them from all harms, on a physical, emotional and spiritual level. This is one of the strongest biological calls to action.

Thus, pediatrics must care and remember that children's health and well-being must be guarded across political borders, across poverty, across starvation. This thesis, and the paper it's so deeply bound to, set themselves to renew this vow.

\section{Intercountry and international adoptees}\label{sec:internationaladoptees}
International adoptees are children with special needs: compared to 19\% of the general population, approximately 39\% of adopted children require special healthcare attention\cite{nelson}. They are of school age, part of a sibling group, members of historically oppressed racial or ethnic groups, or they have considerable physical, emotional, or developmental need\cite{nelson}: all potential elements of vulnerability endangering the child's healthy upbringing. This is not a limited problem: annually more than 30.000 kids are adopted across countries, and, in the United States, of all 136.000 national adoptees in 2008, almost 25\% come from foreign countries; U.S. families adopted 22.884 children in 2004, mostly from China (which accounts for 33\% alone), Ehtiopia, Russia and South Korea (as stated in \cite{nelson}), 8.868 more in 2012 and 4.714 in 2017 (see \cite{usreport}). Further data on annual U.S. international adoptions and their social and financial costs can be found at \cite{usreportsite}.

Although personal experiences obviously vary, most children placed for international adoption have some history of poverty and social hardship in their home countries, and approximately 65\% are adopted from orphanage or institutional settings. As explained in \cite{nelson}, the effects of institutionalization and other early life stresses impact all areas of early growth and development. As a result, many children require specialized support and understanding to overcome such impacts and to reach their full potential. 

Moreover, as in \cite{unreport}, internationally adopted children may withstand a number of juridical and social impairments even after adoption. No generalization can be made on this matter though, since laws and policies greatly differing among countries. They may be stripped of their name (a.e. in Cape Verde, Argentina and Turkey), have no right to inheritance (a.e. in Republic of Moldova and France), see the termination of the relationship with birth parents and relatives (a.e. in Japan, Albania and Togolese Republic), loose their citizenship and not acquire a new one (a.e. Hungary and New Zealand), or even bear limitations on marriage in their adult life (a.e. in Argentina and France). These boundaries are to be considered associated to the emotional and psychological stress of new surroundings, new affections, new habits, and even new climatic environments (which can be clinically relevant, as explained in \ref{par:vitaminddeficiency}). \linebreak All these elements account for some of the factors that contribute to the hardships an adoptee must endure throughout his life and call for a strong action from pediatric physicians and social services employees, as possible support figures which may change these kids' lives forever.


\subsection{Levels and trends in intercountry adoption worldwide}\label{sub:levelsintercountry}
The United Nations Population Division estimates that about 40.000 intercountry adoptions took place each year around 2005, accounting for 15\% of the total number of adoptions (see \cite{unreport}). As shown in Table \ref{tab:intadoptcountriesdestination} and \ref{tab:intadoptcountriesorigin}, the involved countries, both for destination and origin, are relatively few. 

%International adoption countries of destination
\begin{table}[ht]
   \centering
   \begin{tabular}{c l r r l}
      Rank & Receiving country & Number & Percentage & Main country of origin\\
      \hline
      1 & United States of America & 19.056 & 15 & China\\
      2 & France & 3.995 & 90 & Haiti\\
      3 & Spain & 3.951 & 82 & Russia\\
      \textcolor{BrickRed}{\textbf{4}} & \textcolor{BrickRed}{\textbf{Italy}} & \textcolor{BrickRed}{\textbf{2.177}} & \textcolor{BrickRed}{\textbf{68}} & \textcolor{BrickRed}{\textbf{Russia}}\\
      5 & Germany & 1.919 & 34 & Russia\\
      6 & Canada & 1.875 & 46 & China\\
      7 & Sweden & 1.093 & 65 & China\\
      8 & Netherlands & 1.069 & 78 & China\\
      9 & Denmark & 688 & 55 & China\\
      10 & Norway & 664 & 76 & China\\
      11 & Switzerland & 558 & 79 & Colombia\\
      \hline
      \multicolumn{2}{l}{Median} & 370 & 64 &\\
   \end{tabular}
   \captionof{table}{Countries of destination with the largest number of intercountry adoption}
    \source{United Nations Population Division report (see \cite{unreport})}
    \label{tab:intadoptcountriesdestination}
\end{table}

%International adoption countries of destination
\begin{table}[ht]
   \centering
   \begin{tabular}{c l r r l}
      Rank & Receiving country & Number & Percentage & Main receiving country\\
      \hline
      1 & China & 8.644 & 19 & United States\\
      2 & France & 3.995 & 90 & Haiti\\
   
   \end{tabular}
   \captionof{table}{Countries of origin with the largest number of intercountry adoption}
    \source{United Nations Population Division report (see \cite{unreport})}
    \label{tab:intadoptcountriesorigin}
\end{table}

As show in Table \ref{tab:intadoptpercentcountries}, write stuff.

% International adoption percentage countries
\begin{table}[ht!]
    \begin{subtable}[h]{0.45\textwidth}
        \centering
        \begin{tabular}{l l l}
        60 to 74\% & 75 to 89\% & 90\% or more\\
        \hline
        Andorra & Cyprus & Belgium\\
        Australia & Liechtenstein & France\\
        Israel & Netherlands & Luxembourg\\
        Italy & Norway & \\
        Singapore & Spain & \\
        Sweden & Switzerland & \\
        \end{tabular}
        \caption{Receiving Countries}
        \label{tab:receivingcountries}
    \end{subtable}
    \hfill
    \begin{subtable}[h]{0.45\textwidth}
        \centering
        \begin{tabular}{l l l}
        60 to 74\% & 75 to 89\% & 90\% or more\\
        \hline
        Colombia & Georgia & Ethiopia\\
        Latvia & Haiti & Guatemala\\
        Grenada &  & Mali\\
        Honduras &  & Thailand\\
        Niger &  & \\
        Togo &  & \\
        \end{tabular}
        \caption{Countries of origin}
        \label{tab:countriesoforigin}
    \end{subtable}
    \captionof{table}{Countries with the highest percentual international adoptions}
    \source{United Nations Population Division report (see \cite{unreport})}
    \label{tab:intadoptcountries}
\end{table}


Most of these adoptions occur in a few countries. The United States, with over 127,000 adoptions in 2001, accounts for nearly half of all adoptions. Large numbers of adoptions also take place in China (almost 46,000 in 2001) and the Russian Federation (more than 23,000 in 2001). Other countries with sizeable numbers of adoptions are Germany, Ukraine and the United Kingdom, each with over 5,000 adoptions annually. Brazil, Canada, France and Spain also record significant numbers, ranging from 4,000 to 5,000 adoptions per year\cite{unreport}.
The remaining adoptions are distributed among a large number of countries. In 48 of the 118 countries having data on the total number of adoptions, between 100 and 1,000 adoptions occur annually. In another 40 countries, fewer than 100 adoptions take place each year. These countries include Mozambique and Sudan, both of which have large child populations, with at least ten million children each (United Nations, 2005)\cite{unreport}.

Although international adoption is increasingly considered a measure of last resort if the child's birth family or community are unable or unwilling to care for him anymore (see \cite{nelson}), the number of international adoption has steadily been rising in the last couple years, as stated in ... . NOT REALLY
%Add tables from where
%Add tables to where


\subsection{International adoptions in Italy}\label{sub:adoptionsinitaly}
Being a rather open-ended project, i.e. a project in which there is no strict and well-defined set of software requirement specifications, the objectives of the development have been purposefully kept wide and general, as to reflect the idea that the project could follow an exploratory approach.

Nonetheless, there are still some guidelines that have been followed from the beginning to the end of the project:

\begin{itemize}
	\item The project shall result in a working prototype of a Virtual Reality application.
	\item The application shall allow the handling of CFD data; in particular, it shall provide:
	\begin{itemize}
		\item visualization of the data,
		\item interaction with the data,
		\item some basic forms of manipulation of the data.
	\end{itemize}
	\item The application shall allow the import of data from ParaView.
	\item The application shall run compatibly at least on Windows (version 7 or greater), and optionally on Linux.
	\item The application shall support a HTC Vive kit.
	\item The code should be designed to be maintainable, flexible and expandable.
	\item The application should be easy to use, being it aimed at CFD scientists with little to no prior VR experience.
\end{itemize}


\section{Health status and screening protocols}\label{sec:screeningprotocols}
%SEE NELSON, ROLE OF PEDIATRICIAN P. 223
Lorem ipsum dolor sit amet, consectetur adipiscing elit. Pellentesque nibh metus, suscipit a scelerisque sit amet, rhoncus et lectus. Mauris eget erat rutrum, euismod massa id, maximus mauris. Nulla maximus, ex sit amet lacinia consequat, enim ante mollis dui, sit amet tincidunt massa felis id magna. Aenean gravida ante nec volutpat rutrum. Cras eget ullamcorper leo. Curabitur eu volutpat tellus. Integer nec ornare sapien. Fusce ipsum justo, interdum quis libero a, mattis tristique velit. Phasellus rhoncus lorem non ultrices luctus.

\subsection{Health status and screening protocols in Italy}\label{sub:healthstatusandscreeningprotocolsinitaly}
Lorem ipsum dolor sit amet, consectetur adipiscing elit. Pellentesque nibh metus, suscipit a scelerisque sit amet, rhoncus et lectus. Mauris eget erat rutrum, euismod massa id, maximus mauris. Nulla maximus, ex sit amet lacinia consequat, enim ante mollis dui, sit amet tincidunt massa felis id magna. Aenean gravida ante nec volutpat rutrum. Cras eget ullamcorper leo. Curabitur eu volutpat tellus. Integer nec ornare sapien. Fusce ipsum justo, interdum quis libero a, mattis tristique velit. Phasellus rhoncus lorem non ultrices luctus.

\subsection{Health status and screening protocols in the world}\label{sub:healthstatusandscreeningprotocolsinworld}
Lorem ipsum dolor sit amet, consectetur adipiscing elit. Pellentesque nibh metus, suscipit a scelerisque sit amet, rhoncus et lectus. Mauris eget erat rutrum, euismod massa id, maximus mauris. Nulla maximus, ex sit amet lacinia consequat, enim ante mollis dui, sit amet tincidunt massa felis id magna. Aenean gravida ante nec volutpat rutrum. Cras eget ullamcorper leo. Curabitur eu volutpat tellus. Integer nec ornare sapien. Fusce ipsum justo, interdum quis libero a, mattis tristique velit. Phasellus rhoncus lorem non ultrices luctus.

\section{Illnesses and dysfunctions under exam}\label{sec:illnessesanddysfunctions}
Lorem ipsum dolor sit amet, consectetur adipiscing elit. Pellentesque nibh metus, suscipit a scelerisque sit amet, rhoncus et lectus. Mauris eget erat rutrum, euismod massa id, maximus mauris. Nulla maximus, ex sit amet lacinia consequat, enim ante mollis dui, sit amet tincidunt massa felis id magna. Aenean gravida ante nec volutpat rutrum. Cras eget ullamcorper leo. Curabitur eu volutpat tellus. Integer nec ornare sapien. Fusce ipsum justo, interdum quis libero a, mattis tristique velit. Phasellus rhoncus lorem non ultrices luctus.

%DI CIASCUNA INSERISCI EPIDEMIOLOGIA E FISIOPATO

\subsubsection{Infectious diseases}\label{sub:infectiousdiseases}
Lorem ipsum dolor sit amet, consectetur adipiscing elit. Pellentesque nibh metus, suscipit a scelerisque sit amet, rhoncus et lectus. Mauris eget erat rutrum, euismod massa id, maximus mauris. Nulla maximus, ex sit amet lacinia consequat, enim ante mollis dui, sit amet tincidunt massa felis id magna. Aenean gravida ante nec volutpat rutrum. Cras eget ullamcorper leo. Curabitur eu volutpat tellus. Integer nec ornare sapien. Fusce ipsum justo, interdum quis libero a, mattis tristique velit. Phasellus rhoncus lorem non ultrices luctus.

\subsubsection{Blood count disorders and deficiency states}\label{sub:bloodcountdisorders}
Lorem ipsum dolor sit amet, consectetur adipiscing elit. Pellentesque nibh metus, suscipit a scelerisque sit amet, rhoncus et lectus. Mauris eget erat rutrum, euismod massa id, maximus mauris. Nulla maximus, ex sit amet lacinia consequat, enim ante mollis dui, sit amet tincidunt massa felis id magna. Aenean gravida ante nec volutpat rutrum. Cras eget ullamcorper leo. Curabitur eu volutpat tellus. Integer nec ornare sapien. Fusce ipsum justo, interdum quis libero a, mattis tristique velit. Phasellus rhoncus lorem non ultrices luctus.

%INSERISCI LA NORMALITÀ

\paragraph{Iron-deficient anemia}\label{par:iron-deficiency} it can occur.

\paragraph{Vitamin D deficiency}\label{par:vitaminddeficiency} it can occur too.

\subsubsection{Height-weight disorders}\label{sub:heightweightdisorders}
Lorem ipsum dolor sit amet, consectetur adipiscing elit. Pellentesque nibh metus, suscipit a scelerisque sit amet, rhoncus et lectus. Mauris eget erat rutrum, euismod massa id, maximus mauris. Nulla maximus, ex sit amet lacinia consequat, enim ante mollis dui, sit amet tincidunt massa felis id magna. Aenean gravida ante nec volutpat rutrum. Cras eget ullamcorper leo. Curabitur eu volutpat tellus. Integer nec ornare sapien. Fusce ipsum justo, interdum quis libero a, mattis tristique velit. Phasellus rhoncus lorem non ultrices luctus.

%INSERISCI LA NORMALITÀ