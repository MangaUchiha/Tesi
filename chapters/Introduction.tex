%*******10********20********30********40********50********60********70********80

% For all chapters, use the newdefined chap{} instead of chapter{}
% This will make the text at the top-left of the page be the same as the chapter

\chap{Introduction}\label{chap:introduction}
Children's health has historically always been a sensitive and concerning matter for human kind. We can find reason for this in our universal instinctive draw towards the protection and care for our offspring, and in how children can be struck by some of the most devastating and life-wrecking diseases. Sometimes, these are the phenotypical expression of genetical marks, scarred onto and into these kids. Despite the origin, the color of the skin or the culture the child bears in his lineage, human beings feel the need to raise and safeguard them from all harms, on a physical, emotional and spiritual level. This is one of the strongest biological calls to action.

Thus, pediatrics must care and remember that children's health and well-being must be guarded across political borders, across poverty, across starvation. This thesis, and the paper it's so deeply bound to, set themselves to renew this vow.

\section{International Adoptees}\label{sec:internationaladoptees}
International adoptees are children with special needs: compared to 19\% of the general population, approximately 39\% of adopted children require special healthcare attention\cite{nelson}. They are of school age, part of a sibling group, members of historically oppressed racial or ethnic groups, or they have considerable physical, emotional, or developmental need\cite{nelson}: all potential elements of vulnerability endangering the child's healthy upbringing. This is not a limited problem: annually more than 30.000 kids are adopted across countries, and, in the United States, of all 136.000 national adoptees in 2008, almost 25\% come from foreign countries; again in 2012, U.S. families adopted 8.868 children, 22.884 in 2004, mostly from China (33\% alone), Ehtiopia, Russia and South Korea. \cite{nelson}

Although personal experiences obviously vary, most children placed for international adoption have some history of poverty and social hardship in their home countries, and approximately 65\% are adopted from orphanage or institutional settings. As explained in \cite{nelson}, the effects of institutionalization and other early life stresses impact all areas of early growth and development. As a result, many children require specialized support and understanding to overcome such impacts and to reach their full potential. 

Moreover, as in \cite{unreport}, internationally adopted children must withstand a number of juridical and social impairments even after adoption, with laws and policies greatly differing among countries. They may be stripped of their name (a.e. in Cape Verde, Argentina and Turkey), have no right to inheritance (a.e. in Republic of Moldova and France), see the termination of the relationship with birth parents and relatives (a.e. in Japan, Albania and Togolese Republic), loose their citizenship and not acquire the new one (a.e. Hungary and New Zealand), or even bear limitations on marriage in their adult life (a.e. in Argentina and France).

\subsection{Levels and trends in the total number of adoptions}\label{sub:levelsintotal}
Most of these adoptions occur in a few countries. The United States, with over 127,000 adoptions in 2001, accounts for nearly half of all adoptions. Large numbers of adoptions also take place in China (almost 46,000 in 2001) and the Russian Federation (more than 23,000 in 2001). Other countries with sizeable numbers of adoptions are Germany, Ukraine and the United Kingdom, each with over 5,000 adoptions annually. Brazil, Canada, France and Spain also record significant numbers, ranging from 4,000 to 5,000 adoptions per year\cite{unreport}.
The remaining adoptions are distributed among a large number of countries. In 48 of the 118 countries having data on the total number of adoptions, between 100 and 1,000 adoptions occur annually. In another 40 countries, fewer than 100 adoptions take place each year. These countries include Mozambique and Sudan, both of which have large child populations, with at least ten million children each (United Nations, 2005)\cite{unreport}.

\subsection{Levels and trends in intercountry adoption}\label{sub:levelsintercountry}
Although international adoption is increasingly considered a measure of last resort if the child's birth family or community are unable or unwilling to care for him anymore (see \cite{nelson}), the number of international adoption has steadily been rising in the last couple years, as stated in ... .
%Add tables from where
%Add tables to where


If I reference here a piece of code: does it work? \ref{code:nationality} at \pageref{code:nationality}.

Lorem ipsum dolor sit amet, consectetur adipiscing elit. Duis ut congue orci. Cras blandit erat nulla, quis ultrices augue porta a. Ut non ante vel nunc feugiat consequat vel ac ex. Praesent mattis odio et magna laoreet scelerisque. Sed tempus vel ante et volutpat. Nulla pharetra ante nisi, ac tempus sem malesuada non. Integer quis facilisis tellus.

Vivamus et tortor sit amet diam tristique tincidunt quis et sapien. Praesent nec bibendum est. Aenean maximus consectetur elit, et euismod neque aliquet non. Vestibulum ac malesuada magna. Etiam aliquet nec ante ac vulputate. Nullam ut dui tempus, sollicitudin enim in, vestibulum dolor. Sed aliquam elementum nisl rhoncus rutrum. Vestibulum eget arcu non ipsum consequat bibendum non sit amet ligula.

Sed vel auctor urna, vitae consequat ligula. Morbi vel porttitor turpis. Cras ac arcu nulla. Fusce nec posuere nunc. Maecenas et lacus vel sem rhoncus facilisis. Donec vestibulum lorem sit amet tortor finibus dapibus. Duis convallis nisl ac molestie aliquet. Sed ut magna nec lacus pellentesque malesuada. Mauris lacinia vulputate finibus. Aenean est orci, auctor non consequat id, tempor ut ex. Nulla pretium lectus vulputate, rutrum diam non, placerat justo. Aenean mi sapien, viverra sed accumsan at, vehicula aliquam est. Morbi convallis dictum ante in lobortis.

Ut sed dolor orci. Morbi congue elementum suscipit. Proin tempus turpis nec odio euismod fermentum. Praesent ornare dui quis egestas porta. Donec at consectetur orci. Proin ornare convallis libero et feugiat. Quisque sed fringilla justo. Etiam tempus nibh lectus, ut imperdiet ex ultrices tincidunt. Mauris lobortis nulla tortor, non aliquet urna suscipit a. Maecenas non lobortis augue, pulvinar ornare mi. Maecenas euismod nunc lacus, eu ultricies magna rhoncus ut. Donec sit amet sem pretium libero efficitur molestie.

Pellentesque eleifend justo aliquet diam condimentum, accumsan varius lorem pharetra. Nam eu nunc convallis, sodales nisi a, finibus est. Nullam dapibus non tortor eu dapibus. In ut lorem ultrices, blandit dui in, bibendum purus. Aenean finibus non nisl nec maximus. Morbi aliquam tellus eget turpis ultrices, sit amet volutpat felis posuere. Cras a sollicitudin quam. Sed faucibus, ante suscipit iaculis lacinia, metus nisl blandit diam, et egestas nulla lectus sed sapien. Morbi in ex quis leo commodo convallis. Vestibulum diam sapien, finibus a massa sed, gravida fringilla ligula. Curabitur congue odio ut eros suscipit pellentesque. Etiam ut rutrum ipsum. Mauris nunc enim, porttitor at commodo nec, auctor id libero. Maecenas gravida pellentesque felis, ac luctus purus mollis in. Quisque porttitor ultrices nunc id pellentesque.

Suspendisse enim libero, lobortis vitae turpis sed, lacinia dignissim odio. Fusce ultrices scelerisque turpis et lobortis. Integer sapien mauris, luctus sed blandit eget, rutrum vitae dui. Suspendisse imperdiet ornare nibh eget imperdiet. Suspendisse potenti. Donec a elit arcu. Maecenas ac nisi et eros elementum luctus. Donec tempor, nisi ut sagittis laoreet, lorem massa pharetra nulla, vitae consectetur dolor sapien ut lacus. Maecenas a ligula metus. Praesent at augue sem. Quisque faucibus velit vitae tincidunt tempor. Curabitur urna neque, mollis sit amet mi ac, hendrerit tempor ex.

Curabitur in turpis congue, facilisis ligula at, lacinia lectus. Nunc viverra ex sit amet sollicitudin tincidunt. Ut congue iaculis leo, cursus mattis augue elementum sed. Cras varius tortor sed gravida pellentesque. Cras vitae arcu condimentum, feugiat velit a, sagittis ipsum. Donec consequat lobortis lectus et vestibulum. Mauris pharetra tincidunt justo, porta vestibulum arcu mollis id. Nunc euismod lectus nec urna mollis maximus. Nullam ut tortor in nibh luctus feugiat vel sed elit.

Proin tincidunt varius orci. Nunc finibus diam vitae erat suscipit, et vulputate nulla pulvinar. Praesent orci neque, dignissim a fermentum eu, ultricies non ante. Donec ultricies nunc volutpat, sollicitudin sapien sed, imperdiet libero. Mauris finibus, diam quis consectetur ultricies, orci odio dapibus massa, id tristique lectus felis sit amet leo. Sed in tortor pellentesque, laoreet nisi id, imperdiet leo. Sed vehicula dolor at mollis laoreet. Aliquam quis lectus fringilla, ornare turpis vestibulum, faucibus orci. Pellentesque metus velit, iaculis non consequat sit amet, laoreet sit amet est. Suspendisse fringilla viverra risus, ut bibendum mauris dignissim id.

\subsection{International adoptions in Italy}\label{sub:adoptionsinitaly}
Being a rather open-ended project, i.e. a project in which there is no strict and well-defined set of software requirement specifications, the objectives of the development have been purposefully kept wide and general, as to reflect the idea that the project could follow an exploratory approach.

Nonetheless, there are still some guidelines that have been followed from the beginning to the end of the project:

\begin{itemize}
	\item The project shall result in a working prototype of a Virtual Reality application.
	\item The application shall allow the handling of CFD data; in particular, it shall provide:
	\begin{itemize}
		\item visualization of the data,
		\item interaction with the data,
		\item some basic forms of manipulation of the data.
	\end{itemize}
	\item The application shall allow the import of data from ParaView.
	\item The application shall run compatibly at least on Windows (version 7 or greater), and optionally on Linux.
	\item The application shall support a HTC Vive kit.
	\item The code should be designed to be maintainable, flexible and expandable.
	\item The application should be easy to use, being it aimed at CFD scientists with little to no prior VR experience.
\end{itemize}

\subsection{Adoptions in Europe and in the world}\label{sub:adoptionsineurope}
Lorem ipsum dolor sit amet, consectetur adipiscing elit. Pellentesque nibh metus, suscipit a scelerisque sit amet, rhoncus et lectus. Mauris eget erat rutrum, euismod massa id, maximus mauris. Nulla maximus, ex sit amet lacinia consequat, enim ante mollis dui, sit amet tincidunt massa felis id magna. Aenean gravida ante nec volutpat rutrum. Cras eget ullamcorper leo. Curabitur eu volutpat tellus. Integer nec ornare sapien. Fusce ipsum justo, interdum quis libero a, mattis tristique velit. Phasellus rhoncus lorem non ultrices luctus.

\section{Health status and screening protocols}\label{sec:screeningprotocols}
%SEE NELSON, ROLE OF PEDIATRICIAN P. 223
Lorem ipsum dolor sit amet, consectetur adipiscing elit. Pellentesque nibh metus, suscipit a scelerisque sit amet, rhoncus et lectus. Mauris eget erat rutrum, euismod massa id, maximus mauris. Nulla maximus, ex sit amet lacinia consequat, enim ante mollis dui, sit amet tincidunt massa felis id magna. Aenean gravida ante nec volutpat rutrum. Cras eget ullamcorper leo. Curabitur eu volutpat tellus. Integer nec ornare sapien. Fusce ipsum justo, interdum quis libero a, mattis tristique velit. Phasellus rhoncus lorem non ultrices luctus.

\subsection{Health status and screening protocols in Italy}\label{sub:healthstatusandscreeningprotocolsinitaly}
Lorem ipsum dolor sit amet, consectetur adipiscing elit. Pellentesque nibh metus, suscipit a scelerisque sit amet, rhoncus et lectus. Mauris eget erat rutrum, euismod massa id, maximus mauris. Nulla maximus, ex sit amet lacinia consequat, enim ante mollis dui, sit amet tincidunt massa felis id magna. Aenean gravida ante nec volutpat rutrum. Cras eget ullamcorper leo. Curabitur eu volutpat tellus. Integer nec ornare sapien. Fusce ipsum justo, interdum quis libero a, mattis tristique velit. Phasellus rhoncus lorem non ultrices luctus.

\subsection{Health status and screening protocols in the world}\label{sub:healthstatusandscreeningprotocolsinworld}
Lorem ipsum dolor sit amet, consectetur adipiscing elit. Pellentesque nibh metus, suscipit a scelerisque sit amet, rhoncus et lectus. Mauris eget erat rutrum, euismod massa id, maximus mauris. Nulla maximus, ex sit amet lacinia consequat, enim ante mollis dui, sit amet tincidunt massa felis id magna. Aenean gravida ante nec volutpat rutrum. Cras eget ullamcorper leo. Curabitur eu volutpat tellus. Integer nec ornare sapien. Fusce ipsum justo, interdum quis libero a, mattis tristique velit. Phasellus rhoncus lorem non ultrices luctus.

\section{Illnesses and dysfunctions under exam}\label{sec:illnessesanddysfunctions}
Lorem ipsum dolor sit amet, consectetur adipiscing elit. Pellentesque nibh metus, suscipit a scelerisque sit amet, rhoncus et lectus. Mauris eget erat rutrum, euismod massa id, maximus mauris. Nulla maximus, ex sit amet lacinia consequat, enim ante mollis dui, sit amet tincidunt massa felis id magna. Aenean gravida ante nec volutpat rutrum. Cras eget ullamcorper leo. Curabitur eu volutpat tellus. Integer nec ornare sapien. Fusce ipsum justo, interdum quis libero a, mattis tristique velit. Phasellus rhoncus lorem non ultrices luctus.

%DI CIASCUNA INSERISCI EPIDEMIOLOGIA E FISIOPATO

\subsubsection{Infectious diseases}\label{sub:infectiousdiseases}
Lorem ipsum dolor sit amet, consectetur adipiscing elit. Pellentesque nibh metus, suscipit a scelerisque sit amet, rhoncus et lectus. Mauris eget erat rutrum, euismod massa id, maximus mauris. Nulla maximus, ex sit amet lacinia consequat, enim ante mollis dui, sit amet tincidunt massa felis id magna. Aenean gravida ante nec volutpat rutrum. Cras eget ullamcorper leo. Curabitur eu volutpat tellus. Integer nec ornare sapien. Fusce ipsum justo, interdum quis libero a, mattis tristique velit. Phasellus rhoncus lorem non ultrices luctus.

\subsubsection{Blood count disorders and deficiency states}\label{sub:bloodcountdisorders}
Lorem ipsum dolor sit amet, consectetur adipiscing elit. Pellentesque nibh metus, suscipit a scelerisque sit amet, rhoncus et lectus. Mauris eget erat rutrum, euismod massa id, maximus mauris. Nulla maximus, ex sit amet lacinia consequat, enim ante mollis dui, sit amet tincidunt massa felis id magna. Aenean gravida ante nec volutpat rutrum. Cras eget ullamcorper leo. Curabitur eu volutpat tellus. Integer nec ornare sapien. Fusce ipsum justo, interdum quis libero a, mattis tristique velit. Phasellus rhoncus lorem non ultrices luctus.

%INSERISCI LA NORMALITÀ

\subsubsection{Height-weight disorders}\label{sub:heightweightdisorders}
Lorem ipsum dolor sit amet, consectetur adipiscing elit. Pellentesque nibh metus, suscipit a scelerisque sit amet, rhoncus et lectus. Mauris eget erat rutrum, euismod massa id, maximus mauris. Nulla maximus, ex sit amet lacinia consequat, enim ante mollis dui, sit amet tincidunt massa felis id magna. Aenean gravida ante nec volutpat rutrum. Cras eget ullamcorper leo. Curabitur eu volutpat tellus. Integer nec ornare sapien. Fusce ipsum justo, interdum quis libero a, mattis tristique velit. Phasellus rhoncus lorem non ultrices luctus.

%INSERISCI LA NORMALITÀ