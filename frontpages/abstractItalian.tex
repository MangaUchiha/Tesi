% The Abstract Page
\clearpage 
\addtotoc{Abstract (Italian)}  % Add the "Abstract" page entry to the Contents
\abstractitalian{
	%The Thesis Abstract is written here (and usually kept to just this page). 
	%The page is kept centered vertically so can expand into the blank space above the title too \ldots
	
In questo studio sono stati analizzati i dati demografici, clinici e laboratoristici di 285 bambini adottati provenienti dal contesto internazionale, dunque da tutto il mondo, e visitati presso il centro del GNLBI di San Vito al Tagliamento tra gennaio 2002 e dicembre 2017. Tutti i bambini sono stati sottoposti al protocollo diagnostico-assistenziale del GNLBI, il quale include una valutazione clinica completa, esami ematochimici e strumentali, al fine di stimare la prevalenza di malattie infettive, stati carenziali e copertura vaccinale. Le condizioni morbose più frequentemente riscontrate sono state: anemia microcitica (8\%), deficienza di vitamina D (40\%), parassitosi gastroenterica (41\%) e titoli anticorpali vaccinali insufficienti o assenti (65\%). Tutte queste condizioni non sono severe e trattabili. Il 17\% dei bambini esaminati presentava, inoltre, una Mantoux positiva. Le infezioni più severe, come le epatiti B e C, l'HIV e la sifilide si sono riscontrate con estrema rarità, in accordo con la letteratura internazionale specializzata in medicina del bambino adottato e immigrato. 

Inoltre, sono stati analizzati i costi e i benefici di buona parte del protocollo impiegato e si è concluso che l'emocromo, la Mantoux e la titolazione degli anticorpi vaccinali si configurano come esami a costi accettabilmente contenuti e grande impatto sul decorso clinico e terapeutico del bambino. Al contrario, i livelli sierici di vitamina D e gli esami atti a determinare la presenza di infezioni parassitarie gastrointestinali, da HBV, HCV, HIV e \textit{Treponema pallidum} potrebbero non essere impiegati in uno screening di primo livello. Sono stati inoltre valutati l'impatto economico di queste analisi e i potenziali risparmi, se queste fossero abbandonate.

In conclusione, la grande maggioranza dei bambini adottati di provenienza internazionale gode di buona salute e, in questi, sono minimamente rappresentate le patologie mediche maggiori. Nonostante il protocollo diagnostico-assistenziale del GNLBI sia una pietra miliare della pediatria del bambino migrante e adottato, crediamo ci sia spazio per migliorare.
}

