\clearpage 
\addtotoc{Abstract}  % Add the "Abstract" page entry to the Contents
\abstract{
    %The Thesis Abstract is written here (and usually kept to just this page). 
    %The page is kept centered vertically so can expand into the blank space above the title too \ldots
    
We studied the demographic, clinical, and laboratory data from 285 internationally adopted children coming from all over the globe, brought to our attention at the GNLBI Center of \textit{San Vito al Tagliamento}, between January 2002 and December 2017 to evaluate the prevalence of infectious diseases, deficiency states, and immunization status of international adoptees. All children underwent the \textit{GNLBI diagnostic-aiding protocol}, including a physical, laboratory and instrumental examinations. Microcytic anemia (8\%), vitamin D deficiency (40\%), parasitic gastrointestinal infections (41\%), and an incomplete or absent immunization coverage (65\%) were the most frequent condition observed, all of which are treatable and non-life threatening conditions. 17\% of all examined children presented a positive Mantoux test. More severe infections, such as hepatitis B and C, HIV infection, and syphilis were exceptional, in agreement with the most recent medical literature. 

Moreover, we analyzed costs and benefits of sections of the protocol and concluded that blood count, Mantoux test and serum antibody titers of vaccine-preventable disease are highly valuable exams with a direct influence on the therapeutical path of the child. On the other hand, serum vitamin D levels and tests for parasitic gastrointestinal infections, HBV, HCV, HIV, and syphilis may not be so worthwhile. We also studied costs and potential savings.

In conclusion, most adoptive children had good general health, with only a few having major medical problems. Although the \textit{GNLBI diagnostic-aiding protocol} is a milestone of adoption-specialized pediatrics, we believe there is room for improvement from a cost-benefit perspective.
}
