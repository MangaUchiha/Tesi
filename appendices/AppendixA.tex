\chap{Setup Instructions}
This appendix provides the instructions to setup, install and run the software system described in this thesis. They refer to a machine with VR-ready hardware running Windows 10. 

\section{Building ParaView and ParaUnity}
This section provides the instructions for building a working copy of ParaView with the ParaUnity plug-in. It is a simplified and adapted version of the \textit{readme} file of the official ParaUnity repository \cite{paraunity}.

\subsection{Prerequisites}
\begin{itemize}
	\item CMake 3.8.1
	\item Visual Studio 2015 x64 Community Edition
\end{itemize}

\subsection{Obtain the source code}
To obtain a patched, pre-prepared version of the source code for Qt, ParaView and ParaUnity, clone the repository available at \url{https://github.com/vrcranfield/paraviewunity}. Unless specified otherwise, all the following instructions refer to the files contained in this repository.

\subsection{Compile Qt}
The files in \path{Qt 4.8.6} are a patched version of Qt that allows compilation with Visual Studio 2015 x64.

In order to build it do the following:
\begin{enumerate}
	\item Move the content of \path{Qt 4.8.6} in \path{C:\Qt\4.8.6}
	\item Open the \texttt{VS2015 x64 Native Tools Command Prompt} from Start.
	\item cd \path{C:\Qt\4.8.6}
	\item \texttt{./configure.exe -make nmake -platform win32-msvc2015 -prefix \path{C:\Qt\4.8.6} -opensource -confirm-license -nomake examples -nomake tests -nomake demos -debug-and-release}
	\item \texttt{nmake}
	\item \texttt{nmake install}
	\item Add \path{C:\Qt\4.8.6\bin} to the Path environment variable.
\end{enumerate}

\subsection{Compile ParaView}
The files in \texttt{ParaView-v.5.2.0} consist in the official source code of ParaView. 

In order to build it do the following:
\begin{enumerate}
	\item  Open CMake and set source in \path{ParaView-v5.2.0} and build in \path{ParaView-v5.2.0\build}
	\item Configure with ``Visual Studio 14 2015 Win64'' as a generator.
	\item Check that \texttt{PARAVIEW\_QT\_VERSION} is \texttt{4} and that \texttt{QT\_QMAKE\_EXECUTABLE} points to \path{C:\Qt\4.8.6\bin\qmake.exe}. If necessary, configure again.
	\item Generate.
	\item Open with VS2015.
	\item Build solution.
\end{enumerate}

\subsection{Compile ParaUnity}
The files in \texttt{ParaUnity} are the developed and improved version of the official ParaUnity plugin, as described in Chapter.

In order to build ParaUnity do the following:
\begin{enumerate}
	\item Open a terminal in \path{\ParaUnity\Unity3DPlugin}
	\item \texttt{mkdir build}
	\item \texttt{cd build}
	\item \texttt{cmake -G "Visual Studio 14 2015 Win64" -DParaView\_DIR="\path{<PARAVIEW_DIR>\build}" ..}
	\item Open \path{\ParaUnity\Unity3DPlugin\build\Project.sln} in Visual Studio
	\item Right click on the project \texttt{Unity3D}, go to \texttt{C/C++ > Additional Include Directories} and add \path{\verC:\Qt\4.8.6\include\QtNetwork}
	\item Build.
	\item You now have some files (most importantly a \path{Unity3D.dll} file) in \path{\build\Debug}. Remember their location.
\end{enumerate}

\subsection{Loading the plug-in in ParaView}
To load the plug-in in ParaView, do the following:

\begin{enumerate}
	\item Open ParaView 5.2.0 (from \texttt{paraview.exe} in the \path{\build\bin\Debug} folder, or from Visual Studio).
	\item Go to \texttt{Tools > Manage Plugins}, click \texttt{Load New} and locate \texttt{Unity3D.dll}
	\item Open the dropdown entry from Unity3D and select \texttt{Auto Load}.
\end{enumerate}

\section{Building the Unity Application}
This section provides the instruction for obtaining and building a working copy of the Unity Application described in Chapter.

\subsection{Prerequisites}
\begin{itemize}
	\item Unity 5.6.1f1
\end{itemize}

\subsection{Obtain the source code}
To obtain the source code of the Unity Application, clone the repository available at \url{https://github.com/vrcranfield/UnityApplication}. Unless specified otherwise, all the following instructions refer to the files contained in this repository.

\subsection{Compile the application}
In order to build the Unity Application do the following:

\begin{enumerate}
	\item Open the root directory of the project in the Unity editor.
	\item \texttt{File > Build Settings}
	\item Uncheck all scenes apart form the \texttt{Main} scene.
	\item Set \texttt{Target Platform} as \texttt{Windows} and \texttt{Architecture} as \texttt{x86\_64}.
	\item Click build.
	\item Choose the same location as the \texttt{Unity3D.dll} (see previous section).
	\item Save the file as \path{unity_player.exe}
\end{enumerate}

\subsection{Exporting an object from ParaView to Unity}
To test if the system is working correctly, do the following:

\begin{enumerate}
	\item Load any file in ParaView (e.g. a simple sphere)
	\item Click the button with the Unity logo and the \texttt{P}
	\item You should see your Unity scene with the ParaView object in the middle.
\end{enumerate}