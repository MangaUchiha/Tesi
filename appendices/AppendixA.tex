\chap{Code used}\label{chap:codeused}
This appendix provides all the VBA code used in this thesis to elaborate the data set. Further information can be found throughout the thesis, especially in Section \ref{sec:datasetelaboration} at page \pageref{sec:datasetelaboration}.

\section{Age (in months)}\label{sec:ageinmonths}
This VBA expression checks the Age (in months) column and, if it's not empty, it divides it's value by 12, rounding it down, just as age works.

% Excel equation of age (in months)
\begin{minipage}{\linewidth}
\begin{lstlisting}
=IF(
   C2 <> "";
   ROUNDDOWN(
      C2 / 12;
      0
   );
   ""
)
\end{lstlisting}
\end{minipage}

\section{Geographic area of origin}\label{sec:geographicarea}
To further understand how geographic origin influenced the results of our screening program, every nation was grouped up in 4 major continents are areas with the following excel expression.

% Excel equation of nationality
\begin{minipage}{\linewidth}
\begin{lstlisting}
=IF(
   OR(
      E2 = "Russia";
      E2 = "Albania";
      E2 = "Bulgaria";
      E2 = "Hungary";
      E2 = "Ukraine";
      E2 = "Moldavia";
      E2 = "Romania"
   );
   "Eastern Europe";
   IF(
      OR(
         E2 = "Burkina Faso";
         E2 = "Ethiopia";
         E2 = "Ivory Coast";
         E2 = "Congo";
         E2 = "Guinea Bissau";
         E2 = "Africa";
         E2 = "Ghana";
         E2 = "Benin"
      );
      "Africa";
      IF(
         OR(
            E2 = "Colombia";
            E2 = "Brazil";
            E2 = "Guatemala";
            E2 = "Peru";
            E2 = "Costa Rica"
         );
         "South America";
         IF(
            OR(
               E2 = "Armenia";
               E2 = "India";
               E2 = "China";
               E2 = "Vietnam";
               E2 = "Sri Lanka";
               E2 = "Siberia";
               E2 = "Nepal";
               E2 = "Philippines"
            );
            "Asia";
            ""
         )
      )
   )
)
\end{lstlisting}
\end{minipage}

\clearpage %Hack per far vedere il codice proprio qui

\section{Pathological values}\label{sec:pathologicalvalues}
The data set contained numerical values for many laboratory analyses. Cut-off values for these results were established via the most recent literature review, as explained in Sections \ref{sub:cutoffvalues}. In the following sections, the code used to establish which ones where pathological and which were not is displayed and shortly explained.

\subsection{Pathological weight and height}\label{sub:patweightandheight}
These parameters, since they had already been converted to percentile values, were easily implement with the following simple VBA expression:

% Excel equation of pathological weight and height
\begin{minipage}{\linewidth}
\begin{lstlisting}
=IF(
   H2 <> "";
   IF(
      H2 <= 10;
      1;
      0
   );
   ""
)
\end{lstlisting}\
\end{minipage}

\subsection{Pathological hemoglobin}\label{sub:pathemoglobin}
Hemoglobin required a more complicated and sophisticated expression, in order to be stratified, because hemoglobin pathological cut-offs depend on various factors, as described in Section \ref{sub:}. Moreover mild, moderate and severe anemia had to be separated in order to properly evaluate the child's health status; each one had an arbitrary values of 1 (\textit{mild}), 2 (\textit{moderate}) or 3 (\textit{severe}) associated to it.

% Excel equation of pathological hemoglobin
\begin{minipage}{\linewidth}
\begin{lstlisting}
=IF(
   L2 <> "";
   IF(
      AND(
         C2 >= 6;
         C2 < 60
      );
      IF(
         L2 >= 11;
         0;
         IF(
            AND(
               L2 < 11;
               L2 >= 10
            );
            1;
            IF(
               AND(
                  L2 < 10;
                  L2 >= 7
               );
               2;
               3
            )
         )
      );
      IF(
         AND(
            C2 >= 60;
            C2 < 132
         );
         IF(
            L2 >= 11,5;
            0;
            IF(
               AND(
                  L2 < 11,5;
                  L2 >= 11
               );
               1;
               IF(
                  AND(
                     L2 < 11;
                     L2 >= 8
                  );
                  2;
                  3
               )
            )
         );
         IF(
            AND(
               C2 >= 132;
               C2 < 168
            );
            IF(
               L2 >= 12;
               0;
               IF(
                  AND(
                     L2 < 12;
                     L2 >= 11
                  );
                  1;
                  IF(
                     AND(
                        L2 < 11;
                        L2 >= 8
                     );
                     2;
                     3
                  )
               )
            )
         )
      )
   );
   ""
)
\end{lstlisting}\
\end{minipage}





\section{Compile ParaUnity}
The files in \texttt{ParaUnity} are the developed and improved version of the official ParaUnity plugin, as described in Chapter.

In order to build ParaUnity do the following:
\begin{enumerate}
	\item Open a terminal in \path{\ParaUnity\Unity3DPlugin}
	\item \texttt{mkdir build}
	\item \texttt{cd build}
	\item \texttt{cmake -G "Visual Studio 14 2015 Win64" -DParaView\_DIR="\path{<PARAVIEW_DIR>\build}" ..}
	\item Open \path{\ParaUnity\Unity3DPlugin\build\Project.sln} in Visual Studio
	\item Right click on the project \texttt{Unity3D}, go to \texttt{C/C++ > Additional Include Directories} and add \path{\verC:\Qt\4.8.6\include\QtNetwork}
	\item Build.
	\item You now have some files (most importantly a \path{Unity3D.dll} file) in \path{\build\Debug}. Remember their location.
\end{enumerate}

\subsection{Prerequisites}
\begin{itemize}
	\item Unity 5.6.1f1
\end{itemize}
